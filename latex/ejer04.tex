Según se estudió en clase\footnote{Ver también \emph{Numerical Analysis} de R. Burden \& D. Faires, sección sobre interpolación por \emph{splines} cúbicos naturales.}, debemos resolver el sistema $Ax=y$ para encontrar los coeficientes $a_i,b_i,c_i,d_i$, donde
\begin{align*}
	A&=
	\begin{bmatrix}
		1 & 0 & 0 & 0 \\
		h_0 & 2(h_0+h_1) & h_1 & 0 \\
		0 & h_1 & 2(h_1+h_2) & h_2 \\
		0 & 0 & 0 & 1
	\end{bmatrix}\\
	y& = 3
	\begin{bmatrix}
		0\\
		\frac{1}{h_1}(a_2-a_1)-\frac{1}{h_0}(a_1-a_0)\\
		\frac{1}{h_2}(a_3-a_2)-\frac{1}{h_1}(a_2-a_1)\\
		0
	\end{bmatrix}\\
	x&=(c0,c1,c2,c3)^t\\
	h_i&=x_{i+1}-x_{i},\quad i=0,1,2\\
	a_i&=y_i=f(x_i),\quad i=0,1,2,3\\
	b_i&=\frac{1}{h_i}(a_{i+1}-a_i)-\frac{h_i}{3}(2c_i+c_{i+1})\\
	d_i&=\frac{1}{3h_i}(c_{i+1}-c_i)
\end{align*}

Como $c_0=c_3=0$, basta resolver el sistema
\[
	\begin{bmatrix}
		4h & h \\
		h & 4h
	\end{bmatrix}
	\begin{bmatrix}
		c_1\\
		c_2
	\end{bmatrix}
	=
	\frac{3}{h}
	\begin{bmatrix}
		a_2+a_0-2a_1\\
		a_3+a_1-2a_2
	\end{bmatrix}
\]
Note que $h_i=h=0.25$. Sustituyendo $h$ y reacomodando:
\[
\frac{1}{12}
\begin{bmatrix}
1 & 1/4 \\
1/4 & 1
\end{bmatrix}
\begin{bmatrix}
c_1\\
c_2
\end{bmatrix}
=
\begin{bmatrix}
a_2+a_0-2a_1\\
a_3+a_1-2a_2
\end{bmatrix}
\]
Si se resuelve este sistema en MATLAB, se obtiene que $c_1\approx59.1967224132176$ y $c_2\approx100.2945081600630$.

Ahora aproximo el valor de $f(2.65)$. Puesto que $2.5<2.65<2.76$ entonces debemos emplear el polinomio del tercer subintervalo, $s_2(x)=a_2+b_2(x-x_2)+c_2(x-x_2)^2+d_2(x-x_2)^3$. Usando las fórmulas anteriores,
\begin{align*}
	a_2&\approx79.186210744572577\\
	b_2&\approx120.1757303031561\\
	c_2&\approx100.2945081600630\\
	d_2&\approx-133.7260108800840
	\intertext{y entonces}
	s_2(2.65)&\approx99.017871436927123
\end{align*}

El valor que se obtiene en una calculadora es $f(2.65)\approx98.370568585361809$. El error es $|s_2(2.65)-f(2.65)|\approx0.647302851565314$ y el error relativo es
\[
	\frac{|s_2(2.65)-f(2.65)|}{|f(2.65)|}\approx0.006580249162671
\]
